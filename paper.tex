\documentclass{article}

\usepackage{amsmath,amssymb,mathtools}
\usepackage{hyperref}
\usepackage[round]{natbib}   % omit 'round' option if you prefer square brackets

%%%%%%%%%%%%%%%%%%%%%%%%%%%%%%%%%%%%%%%%%%%%%%%%%%%%%%%%%%%%%%%%%%%%%%%%%%%%%%%%

\DeclareMathOperator*{\argmin}{arg\,min}
\DeclareMathOperator*{\argmax}{arg\,max}

\newcommand{\loss}{\ell}
\newcommand{\sloss}{S}
\newcommand{\slossave}{\sloss^{\textit{ave}}}
\newcommand{\slossmax}{\sloss^{\textit{max}}}
\newcommand{\slossvar}{\sloss^\textit{var}}
\newcommand{\slossvarp}{\sloss^\textit{varp}}

\newcommand{\x}{x}
\newcommand{\xstar}{\x^*}
\newcommand{\xhat}{\hat\x}
\newcommand{\xhats}{\xhat^s}
\newcommand{\xhatn}{\xhat^{\textit{naive}}}
\newcommand{\xhato}{\xhat^{\textit{oracle}}}
\newcommand{\xhatr}{\xhat^{\textit{rational}}}
\newcommand{\V}{V}
\newcommand{\Vhat}{\hat\V}

\newcommand{\T}{T}
\newcommand{\Tfptp}{\T^{\textit{fptp}}}

\newcommand{\D}{\mathbb{D}}
\newcommand{\R}{\mathbb{R}}
\newcommand{\Rnn}{\R_{\ge0}}

%%%%%%%%%%%%%%%%%%%%%%%%%%%%%%%%%%%%%%%%

\usepackage{amsthm}
\makeatletter
\def\th@definition{%
  \thm@notefont{}% same as heading font
  \normalfont % body font
}
\makeatother
\newtheorem{theorem}{Theorem}
\newtheorem{lemma}{Lemma}
\newtheorem{conj}{Conjecture}
\theoremstyle{definition}
\newtheorem{problem}{Problem}
\newtheorem{defn}{Definition}
\newtheorem{note}{Note}
\newtheorem{example}{Example}


%%%%%%%%%%%%%%%%%%%%%%%%%%%%%%%%%%%%%%%%%%%%%%%%%%%%%%%%%%%%%%%%%%%%%%%%%%%%%%%%

\begin{filecontents}{paper.bib}
@article{feddersen1996swing,
  title={The swing voter's curse},
  author={Feddersen, Timothy J and Pesendorfer, Wolfgang},
  journal={The American economic review},
  pages={408--424},
  year={1996},
  publisher={JSTOR}
}

@article{edlin2007voting,
  title={Voting as a rational choice: Why and how people vote to improve the well-being of others},
  author={Edlin, Aaron and Gelman, Andrew and Kaplan, Noah},
  journal={Rationality and society},
  volume={19},
  number={3},
  pages={293--314},
  year={2007},
  publisher={Sage Publications Sage UK: London, England}
}

@article{fowler2007beyond,
  title={Beyond the self: Social identity, altruism, and political participation},
  author={Fowler, James H and Kam, Cindy D},
  journal={Journal of Politics},
  volume={69},
  number={3},
  pages={813--827},
  year={2007},
  publisher={Wiley Online Library}
}

@article{gerber2009descriptive,
  title={Descriptive social norms and motivation to vote: Everybody's voting and so should you},
  author={Gerber, Alan S and Rogers, Todd},
  journal={The Journal of Politics},
  volume={71},
  number={1},
  pages={178--191},
  year={2009},
  publisher={Cambridge University Press New York, USA}
}

@article{feddersen2009moral,
  title={Moral bias in large elections: theory and experimental evidence},
  author={Feddersen, Timothy and Gailmard, Sean and Sandroni, Alvaro},
  journal={American Political Science Review},
  volume={103},
  number={02},
  pages={175--192},
  year={2009},
  publisher={Cambridge Univ Press}
}

@article{kroneberg2010norms,
  title={Norms and rationality in electoral participation and in the rescue of Jews in WWII: An application of the model of frame selection},
  author={Kroneberg, Clemens and Yaish, Meir and Stock{\'e}, Volker},
  journal={Rationality and Society},
  volume={22},
  number={1},
  pages={3--36},
  year={2010},
  publisher={Sage Publications Sage UK: London, England}
}

@article{gelman2012probability,
  title={What is the probability your vote will make a difference?},
  author={Gelman, Andrew and Silver, Nate and Edlin, Aaron},
  journal={Economic Inquiry},
  volume={50},
  number={2},
  pages={321--326},
  year={2012},
  publisher={Wiley Online Library}
}

@article{mcmurray2012aggregating,
  title={Aggregating information by voting: The wisdom of the experts versus the wisdom of the masses},
  author={McMurray, Joseph C},
  journal={The Review of Economic Studies},
  pages={rds026},
  year={2012},
  publisher={Oxford University Press}
}
\end{filecontents}
\immediate\write18{bibtex paper}

%%%%%%%%%%%%%%%%%%%%%%%%%%%%%%%%%%%%%%%%%%%%%%%%%%%%%%%%%%%%%%%%%%%%%%%%%%%%%%%%

\title{Rational Altruism}
\author{Mike Izbicki}
\begin{document}
\maketitle

%%%%%%%%%%%%%%%%%%%%%%%%%%%%%%%%%%%%%%%%%%%%%%%%%%%%%%%%%%%%%%%%%%%%%%%%%%%%%%%%

\section{Voting}

Let $P=\{p_1,...,p_n\}$ be a set of $n$ people.
Let $X$ be a set of choices.%
\footnote{
    We are making no assumptions about the set of choices.
    For example, it may by finite, countably infinite, or uncountable.
    Similarly, it may have no special structure, or it may have an associated topology.
}
Each person $p_i$ has an associated loss function $\loss_i : X \to \Rnn$.
The preferred choice for person $i$ is then given by
\begin{equation}
\xstar_i = \argmin_{\x\in X} \loss_i(\x)
.
\end{equation}
In a voting scenario, each person selects some choice.
We will denote the choice of person $i$ using strategy $s$ as $\xhats_i$.
The naive voting strategy is to select the choice that corresponds to your minimum loss.
That is,
\begin{equation}
\xhatn_i = \xstar_i
.
\end{equation}
Arrow's impossibility theorem shows that this is a suboptimal strategy in many practical scenarios.

To discuss better voting strategies, we need to first discuss how votes determine a choice.
In particular, society selects some vote aggregation function $\T : \{P\times X\} \to X$. 
This function takes as input the votes of the population and returns a choice.
The most common aggregation function is first-past-the-post (also known as majority rules).
To consider the optimal voting strategy, let us first assume that we have full knowledge of everyone else's vote.
In particular, denote by $V_i : \{P\times X\}$ the set of all votes except the vote of person $i$.
Then the oracle voting strategy is
\begin{equation}
\xhato_i = \argmin_{\x\in X} \loss_i(T(V_i\cup\{(p_i,\x)\}))
.
\end{equation}
The oracle strategy obviously cannot actually be implemented in practice because the set $V_i$ is unknowable.

A realistic strategy would be to approximate the set $V_i$ using our knowledge of how other people are likely to vote.
We begin by assuming all voters are perfectly rational and have unbounded computational power.
Denote by 
\begin{equation}
\begin{split}
\xhatr_i{}^{,j} &= \argmin_{\x\in X} \loss_i(T(\Vhat^{j-1} \cup \{(p_i,\x_i)\})) \\
\Vhat^0 &= \{(p_i,x_i) : i \in [n]\} \\
\Vhat^j &= \{(p_i,\xhatr_i{}^{,j}) : i \in [n]\}
\end{split}
\end{equation}

\begin{conj}
Under reasonable assumptions, the sequence $\Vhat^j$ converges.
The converged solution is pareto optimal.
\end{conj}
In reality, most voters will not act rationally.
We ignore this for now.

%%%%%%%%%%%%%%%%%%%%%%%%%%%%%%%%%%%%%%%%

\section{Social good}

For each person $i$, denote by $\lambda_i : X \to \Rnn$ the intrinsic loss of a choice.
The intrinsic losses of a person are not allowed to depend on any other person.
We can now define social loss functions.
Here are some examples:

\begin{align}
\slossmax(x) &= \max \{ \lambda_p (\x) : p \in P \}
\\
\slossave(x) &= \frac{1}{n}\sum_{p\in P} \lambda_p (\x)
\\
\slossvar(x) &= \frac{1}{n}\sum_{p\in P} (\lambda_p(\x) - \slossave(x))^2
\\
\slossvarp(x) &= \frac{1}{n}\sum_{p\in P} (\max\{\lambda_p(\x) - \slossave(x),0\})^2
\end{align}

%Let $P$ be divided into two disjoint sets $P_1$ and $P_2$.
%
%\begin{equation}
%(x) = 
%\end{equation}

%\begin{equation}
%S_\gamma(x) = \sloss(x) + \gamma \left( \slossmax(x) - \slossave(x) \right)
%\end{equation}

An altruist would decide to minimize some social loss function instead of their own personal loss function.
That is the altruist's loss $\loss_i = S$.

%%%%%%%%%%%%%%%%%%%%%%%%%%%%%%%%%%%%%%%%

\section{Randomness}

Let $\D$ be the space of univariate probability distributions with support on $\Rnn$.
We now generalize the above discussion so that the loss function has type $\loss_i : X \to \D$.
The intended interpretation is that the loss of a given choice is no longer fixed;
instead it follows some distribution.
This distribution may capture inherent randomness in the process,
or if the process is non-stochastic it might capture our uncertainty about the problem.
In this setting, there is no one preferred choice.
There is a family of possible choices that sit along a bias-variance curve.

As before, assume all votes except for person $i$ are fixed.
Then
\begin{equation}
%\xhato_i = \argmin_{\x\in X} \loss_i(T(V_i\cup\{(p_i,\x)\}))
\sloss(\T(\V_i)) - \sloss(\T(\V_i\cup\{(p_i,\x)\}))
\end{equation}
is now a random variable.

%%%%%%%%%%%%%%%%%%%%%%%%%%%%%%%%%%%%%%%%%%%%%%%%%%%%%%%%%%%%%%%%%%%%%%%%%%%%%%%%

\clearpage

\section{References}

Condorcet provide an argument that voting is an optimal decision making strategy in the 1700s.

\cite{feddersen1996swing} present the swing voter's curse.
That is, less informed voters can strictly prefer to abstain from voting even when voting is costless and they have a strict preference of candidates.
\cite{mcmurray2012aggregating} builds on this work to show that in equilibrium, the number of people who choose to vote is optimal.
Furthermore, when someone becomes more informed, they cause other people to become less likely to vote.

Is auction theory relevant?

\cite{edlin2007voting} makes the case that non-selfish behavior motivates voters, and this results in good voter turnout.

\cite{fowler2007beyond} discuss a model that distinguishes ``altruistic'' social interactions and ``identity'' social actions.

\cite{gerber2009descriptive} present empirical evidence that discussing low voter turnout rates causes lower turnout rates, and talking about high turnout rates causes higher turnout rates.
The resulting conclusion is that people decide to vote/not vote similarly to how other people decide to.

\cite{feddersen2009moral} claims that larger elections cause the participants to behave more morally.
They provide both a theoretical model and experiments.

\cite{kroneberg2010norms} argue that norms are a major cause of voter turnout.
They present a psychological model called the Model of Frame Selection to justify their claim.

\cite{gelman2012probability} calculate the probability that an individual vote makes a difference in the 2008 US election based on survey data.
\bibliographystyle{plainnat}
\bibliography{paper}

\end{document}
